\documentclass[11pt,a4paper]{article}

\usepackage[left=2cm,text={17cm,24cm},top=3cm]{geometry}
\usepackage[english]{babel}
\usepackage[utf8]{inputenc}
\usepackage[T1]{fontenc}

\usepackage{url}
\usepackage{float}
\usepackage{comment}
\usepackage{etoolbox}
\usepackage{graphicx}
\usepackage{hyperref}
\usepackage{tocloft}

\def\UrlBreaks{\do\/\do-\do\&\do=\do\_\do?} % URL breaking characters

\newcommand{\red}[1]{\textcolor{red}{#1}} % \red{text in red}
\newcommand{\blue}[1]{\textcolor{blue}{#1}} % \blue{text in blue}
\newcommand{\TODO}{\textbf{\textcolor{red}{TODO}}} % red bold TODO
\newcommand{\tilda}{\raisebox{0.5ex}{\texttildelow}} % command \tilda for '~' character

\graphicspath{{img/}} % path to images
\patchcmd{\thebibliography}{\section*{\refname}}{}{}{} % do not create section for bibliography
\hypersetup{
    linktoc    = all,
    colorlinks = true,
    citecolor  = green,
    linkcolor  = red,
    urlcolor   = blue,
}

\begin{document}

\begin{titlepage}

    \begin{center}
        % FIX: lines must end with '%', if not then it will result in an incorrect centering
        \vfill {%
            \Huge{%
                \textsc{%
                    Faculty of Informatics\\[3mm]%
                    Masaryk University%
                }%
            }%
        }%

        \hfill\\[15mm]

        \begin{figure}[!h]
            \centering
            \includegraphics[scale=3]{img/muni-fi-logo.pdf}
        \end{figure}

        \hfill\\[10mm]

        \Huge{
            \textbf{
                IV064
            }
        }

        \hfill\\[-10mm]

        \huge{
            \textbf{
                Information Society
            }
        }

        \hfill\\[10mm]

        \LARGE{
            \textbf{
                Open access to source code, and results, open systems
            }
        }
        \vfill

    \end{center}

        \Large{
            \noindent Adrián Tóth (491322)\hfill \today
        }

\end{titlepage}

\setlength{\parskip}{0pt}
    {
        \hypersetup{
            hidelinks=true
        }
        \tableofcontents
    }
\setlength{\parskip}{0pt}

\newpage

\section{The Open Access}

    Nowadays, the open access term is used in many ways. Firstly, let's declare the meaning of this expression within the context of IT to better understand other things that will be mentioned later. You may heard the phrases like '\textit{Open access to source code}' as well as '\textit{Open source software}'. All of these names indicates one attribute of the software development - the right of free access to the source code of a software to anyone. Open source indicates the free way of the software source code distribution that can be restricted by further open source licenses such as GPL - GNU General Public License, which will be described later.

\section{My Notes}

    \TODO\cite{BOOK}

    \begin{itemize}
      \item \url{https://ieeexplore.ieee.org/document/4556977}\\[-8mm]
      \item \url{https://ieeexplore.ieee.org/document/5662568}\\[-8mm]
      \item \url{https://ieeexplore.ieee.org/document/1620054}\\[-8mm]
    \end{itemize}

\newpage

\section{References}
\bibliographystyle{englishiso}
\begin{flushleft}
    \bibliography{quotation}
\end{flushleft}

\end{document}
